\begin{tcolorbox}[colback=gray!50,enhanced,sharp corners,frame hidden,halign=left]
    \textbf{\textit{QUESTION 1 - FORCE B}}
\end{tcolorbox}

a) State the Fourier Series representation of Force B. Provide your FULL working (handwritten and scanned or typed).
\begin{flushright}
    [10 Marks*]
\end{flushright}

% insert workings here 






\noindent\makebox[\linewidth]{\rule{\linewidth}{0.4pt}}
b) Insert your figure here. \\
\textit{Plot (on the same figure) Force B and a sequence of approximations containing more and more terms.}
\begin{flushright}
    [3 Marks$^{\wedge}$]
\end{flushright}

% insert workings here








\noindent\makebox[\linewidth]{\rule{\linewidth}{0.4pt}}
c) Insert discussion of results here. \\
\textit{Discuss the result in b). How does the solution change as more terms are added? How many terms should we choose for an accurate representation of Force B?}
\begin{flushright}
    [2 Marks$^{\wedge}$]
\end{flushright}

% insert workings here



\noindent\makebox[\linewidth]{\rule{\linewidth}{0.4pt}}
Insert your MATLAB code here:

\begin{lstlisting}
%% some matlab code
1 + 1 == 2
\end{lstlisting}


\clearpage