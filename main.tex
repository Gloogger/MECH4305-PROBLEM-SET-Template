\documentclass[11pt]{article}
% font setup, use Lualatex to compile!
\usepackage{fontspec}
%\setmainfont{Times New Roman}
% change caption as I WANT
\usepackage{caption}
% add delete line
\usepackage[normalem]{ulem}
% page margin = 25.4mm on each side
\usepackage[inner=2.54cm,outer=2.54cm]{geometry}
%\setmainfont{Calibri}
\setmainfont[
 ExternalLocation=font/,
 BoldFont={CalibriBold}, 
 ItalicFont={CalibriItalic},
 BoldItalicFont={CalibriBoldItalic}
 ]{Calibri}
% initialize image referencing
\usepackage{graphicx}
\graphicspath{ {images/} }
% allows the insertion of figures as A3 or A2 pages
\usepackage[paper=A4,pagesize]{typearea}
\usepackage{afterpage}
% insert pdf pages
\usepackage{pdfpages}
% allows really loooooonnnnngggg table
\usepackage{longtable}
% allows landscape figure
\usepackage{rotating}
\usepackage{pdflscape}
\usepackage{lscape}
\usepackage{float}
% initialize clickable references
\usepackage[unicode]{hyperref}
\hypersetup{
    colorlinks,
    citecolor=black,
    filecolor=black,
    linkcolor=black,
    urlcolor=black
}
% allow multirow and multicolumn in tables
\usepackage{multirow}
% reference styles
\usepackage[style=ieee, natbib=true]{biblatex}
\addbibresource{refs.bib}
% appendices
\usepackage{appendix}
% scale large matrix, use \scalemath{}
\newcommand\scalemath[2]{\scalebox{#1}{\mbox{\ensuremath{\displaystyle #2}}}}
% initialize math tools
\usepackage{bm}
\usepackage{amsmath}
\usepackage{cancel}
\usepackage{amssymb}
\usepackage{mathtools}
\usepackage{empheq}
\usepackage{arydshln}
\newcommand\numberthis{\addtocounter{equation}{1}\tag{\theequation}}
% define sign convention
\makeatletter

\newcommand*\curveplus{%
  \mathbin{\rotatebox[origin=c]{90}{$\m@th\curvearrowleft$}+}}

\newcommand*\rightplus{%
  \mathpalette\@rightplus\relax}
\newcommand*\@rightplus[1]{%
  \mathbin{\vcenter{\hbox{$\m@th\overset{#1+}{\to}$}}}}

\newcommand*\upplus{%
  \mathbin{+\mathord\uparrow}}

\makeatother
% add indentation
\usepackage{tocloft}
% remove paragraph indentations and add newlines between paragraphs
\usepackage[parfill]{parskip}

%underline
\usepackage{soul}

% Code Block
% load package with ``framed'' and ``numbered'' option.
\usepackage[framed,numbered,autolinebreaks,useliterate]{mcode}

\usepackage{geometry}
% colored background
\usepackage[most]{tcolorbox}
% enable the initial name field
\usepackage{calc}
\newlength{\mylen}
\setlength{\mylen}{\widthof{Address for Communication}}
\newcommand{\details}[2]{%
\makebox[\mylen][l]{#1} \,: #2 \\
}

\begin{document}
    \begin{titlepage}
        \begin{tcolorbox}[colback=gray!50,enhanced,sharp corners,frame hidden,halign=center]
        \Large\textbf{MECH4305 PROBLEM SET 1\\ Fourier Analysis and Convolution}
        \end{tcolorbox}

        \textit{\textbf{Student Details}}
        
        \vspace{0.15cm}
        
        \noindent
        \details{First Name}{Jack}
        \details{Family Name}{Sparrow}
        \details{ZID}{z1234567}
        
        \vspace{0.3cm}
        
        \textbf{Total marks available = 70}
        
        \vspace{0.5cm}
        
        Marks are included in the template. \\
        * indicates full working must be included to receive these marks. Full working must be legible. \\
        $^{\wedge}$ indicates full MATLAB code must be included to receive these marks.
        
        \vspace{0.5cm}
        
        Figures need to be presented to a professional standard with:
        \begin{itemize}
            \item Axes labelled with units
            \item Legend included where appropriate
            \item Appropriate line styles and widths
            \item Readable figure font style and size
        \end{itemize}

        \vspace{0.5cm}
        
        Suggested method for inserting your hand-written working into this template:
        \begin{itemize}
            \item Download the Office Lens app to your mobile device
            \item Scan your work
            \item Save to OneDrive 
            \item Insert your working into this template or merge your working with this template (in pdf)
        \end{itemize}

        \vspace{0.5cm}

        Upload your solution in pdf to the Moodle submission box.


    \end{titlepage}
    
    \input{chapters/Q1_Force_A.tex}
    \label{sec:Q1_Force_A}
    \pagenumbering{arabic}
   
    \begin{tcolorbox}[colback=gray!50,enhanced,sharp corners,frame hidden,halign=left]
    \textbf{\textit{QUESTION 1 - FORCE B}}
\end{tcolorbox}

a) State the Fourier Series representation of Force B. Provide your FULL working (handwritten and scanned or typed).
\begin{flushright}
    [10 Marks*]
\end{flushright}

% insert workings here 






\noindent\makebox[\linewidth]{\rule{\linewidth}{0.4pt}}
b) Insert your figure here. \\
\textit{Plot (on the same figure) Force B and a sequence of approximations containing more and more terms.}
\begin{flushright}
    [3 Marks$^{\wedge}$]
\end{flushright}

% insert workings here








\noindent\makebox[\linewidth]{\rule{\linewidth}{0.4pt}}
c) Insert discussion of results here. \\
\textit{Discuss the result in b). How does the solution change as more terms are added? How many terms should we choose for an accurate representation of Force B?}
\begin{flushright}
    [2 Marks$^{\wedge}$]
\end{flushright}

% insert workings here



\noindent\makebox[\linewidth]{\rule{\linewidth}{0.4pt}}
Insert your MATLAB code here:

\begin{lstlisting}
%% some matlab code
1 + 1 == 2
\end{lstlisting}


\clearpage
    \label{sec:Q1_Force_B}
    
    \begin{tcolorbox}[colback=gray!50,enhanced,sharp corners,frame hidden,halign=left]
    \textbf{\textit{QUESTION 2}}
\end{tcolorbox}

a) State the Fourier Series representation of the force. Provide your FULL working (handwritten and scanned or typed).
\begin{flushright}
    [8 Marks*]
\end{flushright}

% insert workings here 






\noindent\makebox[\linewidth]{\rule{\linewidth}{0.4pt}}
b) State the steady state response of the machine assuming it can be represented as an undamped mass-spring system. Provide your FULL working (handwritten and scanned or typed).
\begin{flushright}
    [6 Marks*]
\end{flushright}

% insert workings here 


\noindent\makebox[\linewidth]{\rule{\linewidth}{0.4pt}}
c) Insert your figure here. \\
\textit{Plot the steady state response of the machine from t = 0 to 10 s with n = 20.}
\begin{flushright}
    [3 Marks$^{\wedge}$]
\end{flushright}

% insert workings here 


\noindent\makebox[\linewidth]{\rule{\linewidth}{0.4pt}}
Insert your MATLAB code here:

\begin{lstlisting}
%% some matlab code
1 + 1 == 2
\end{lstlisting}


\clearpage
    \label{sec:Q2}
    
    \begin{tcolorbox}[colback=gray!50,enhanced,sharp corners,frame hidden,halign=left]
    \textbf{\textit{QUESTION 3 - Force A}}
\end{tcolorbox}

a) State the response of a mass-spring system to Force A using the convolution integral. Provide your FULL working (handwritten and scanned or typed).
\begin{flushright}
    [4 Marks*]
\end{flushright}

% insert workings here 






\noindent\makebox[\linewidth]{\rule{\linewidth}{0.4pt}}
b) Insert your figure here. \\
\textit{Find the response using a numerical integrator such as MATLAB’s ode45 and compare the integral and numerical solutions on a plot for t = 0 to 10 s. }
\begin{flushright}
    [4 Marks$^{\wedge}$]
\end{flushright}

% insert workings here 








\noindent\makebox[\linewidth]{\rule{\linewidth}{0.4pt}}
c) Insert your figure here. \\
\textit{Repeat the numerical solution in b) with a damping constant $\zeta = 0.3$.}
\begin{flushright}
    [3 Marks$^{\wedge}$]
\end{flushright}

% insert workings here 








\noindent\makebox[\linewidth]{\rule{\linewidth}{0.4pt}}
d) Insert discussion of results here. \\
\textit{Describe the effect of damping on the response to Force A.}
\begin{flushright}
    [1 Marks$^{\wedge}$]
\end{flushright}

% insert workings here 









\noindent\makebox[\linewidth]{\rule{\linewidth}{0.4pt}}
Insert your MATLAB code here:

\begin{lstlisting}
%% some matlab code
1 + 1 == 2
\end{lstlisting}


\clearpage




    \label{sec:Q3_Force_A}
    
    \input{chapters/Q3_Force_B.tex}
    \label{sec:Q3_Force_B}

\end{document}
